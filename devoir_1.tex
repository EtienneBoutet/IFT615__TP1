\documentclass{article}
\usepackage[left=1in, right=1in, top=1in, bottom=1in]{geometry}
\usepackage[utf8]{inputenc}
\usepackage[english,frenchb]{babel}
\usepackage{graphicx}
\usepackage{algorithm}
\usepackage{hyperref}
\usepackage{enumitem}

\newenvironment{itemize*}%
  {\vspace*{.1cm}\begin{itemize}%
      \setlength{\itemsep}{0pt}%
      \setlength{\parskip}{0pt}}%
  {\end{itemize}\vspace*{.1cm}}

\begin{document}
\title{IFT 615: Devoir 1\\ Travail individuel}
\author{Remise: 26 mai 2017, 17h00 ({\bf au plus tard})}
\date{}
\maketitle

Ce devoir comporte 3 questions de programmation. Vous trouverez tous les fichiers
nécessaires pour ce devoir ici: \url{https://www.dropbox.com/s/w6s3301jubcraj6/devoir_1.zip?dl=0}

\vspace*{.3cm}
Veuillez soumettre vos solutions à l'aide de l'outil web {\bf \href{http://opus.dinf.usherbrooke.ca/turnin_off}{turnin}}. Les fichiers suivants sont requis:

\begin{verbatim}
solution_taquin.py solution_tictactoe.py solution_connect4.py
\end{verbatim}


\begin{enumerate}[itemsep=20pt]
\item {\bf [5 points]} Programmez l'algorithme A* pour résoudre le jeu
  du taquin à 3$\times$3 cases ({\em 8-puzzle}: \url{http://fr.wikipedia.org/wiki/Taquin}).

  Le programme doit être dans le langage Python. Plus spécifiquement,
  vous devez remettre un fichier {\tt solution\_taquin.py} contenant une
  fonction nommée {\tt joueur\_taquin} prenant 4
  arguments (dans l'ordre):
  \begin{itemize*}
  \item {\tt etat\_depart}: Objet de la classe {\tt TaquinEtat} indiquant
    l'état initial du jeu.
  \item {\tt fct\_estEtatFinal}: Fonction qui prend en entrée un objet
    de la classe {\tt TaquinEtat} et qui retourne {\tt True} si l'état
    passé en paramètre est l'état final.
  \item {\tt fct\_transitions}: Fonction qui prend en entrée un objet
    de la classe {\tt TaquinEtat} et qui retourne la liste des états
    voisins (objets {\tt TaquinEtat}) pour l'état donné.
    {\bf On suppose que le coût de chaque transition est de 1}.
  \item {\tt fct\_heuristique}: C'est la fonction heuristique à
    utiliser.  Cette fonction prend en entrée un objet
    {\tt TaquinEtat} et retourne l'heuristique pour cet état.
  \end{itemize*}

  Votre fonction {\tt joueur\_taquin} doit retourner la solution au
  jeu, c'est-à-dire la liste des objets {\tt TaquinEtat} correspondant
  à la solution trouvée par A* (de l'état de départ vers l'état final
  inclusivement, dans l'ordre).

  Un fichier initial {\tt solution\_taquin.py} vous est fourni. Il
  contient entre autres la définition d'une classe {\tt AEtoileTuple},
  qui joue le rôle des triplets $(n,f(n),parent)$ utilisés par A* dans ses
  listes {\it open} et {\it closed}. Cette classe a les méthodes
  suivantes:
  \begin{itemize*}
  \item {\tt \_\_init\_\_(self, etat, f, parent=None)}: Constructeur
    ayant comme argument {\tt etat} (objet {\tt
      TaquinEtat}) qui correspond à un état $n$, {\tt f} (nombre à virgule
    flottante) qui correspond à la valeur de la fonction d'évaluation $f(n)$ pour cet état, ainsi que {\tt parent} (objet
    {\tt AEtoileTuple}) correspondant au parent de l'état $n$ dans la
    recherche (avec {\tt None} comme valeur par défaut, appropriée pour la racine de
    l'arbre de recherche).  Une fois un objet {\tt AEtoileTuple}
    construit, ces arguments sont accessibles comme des champs de
    l'objet ({\tt objet.etat}, {\tt objet.f} et {\tt objet.parent}).
  \item {\tt \_\_eq\_\_(self, autre)}: Retourne {\tt True} si l'objet {\tt
      autre} (aussi de la classe {\tt AEtoileTuple}) correspond au
    même état du jeu. Cette méthode est invoquée lors des tests
    d'égalité ({\tt ==}) entre deux objets. Elle est aussi invoquée
    implicitement lors des tests d'appartenance ({\tt in}).
    Ceci permet d'avoir le comportement suivant: si {\tt atuple} est
    un objet {\tt AEtoileTuple} et que {\tt open} est une liste
    d'objets {\tt AEtoileTuple}, alors le test {\tt atuple in open}
    retournera {\tt True} si {\tt open} contient un objet {\tt
      AEtoileTuple} ayant le même état que {\tt atuple}.
  \item {\tt \_\_ne\_\_(self, autre)}: Retourne l'inverse de {\tt \_\_eq\_\_(self, autre)}.
  \item {\tt \_\_cmp\_\_(self, autre)}: Retourne la différence entre
    la valeur {\tt f} de la fonction d'évaluation et celle de l'objet {\tt
      autre} (aussi de la classe {\tt AEtoileTuple}).
      Ceci permet d'avoir le comportement suivant: si {\tt atuple} est
    un objet {\tt AEtoileTuple} et que {\tt open} est une liste
    d'objets {\tt AEtoileTuple}, alors l'opération {\tt min(open)}
    retournera l'objet {\tt AEtoileTuple} ayant le plus petit {\tt f}.
  \end{itemize*}

  À noter que {\tt parent} doit pointer vers un objet {\tt AEtoileTuple}
  et non vers un état (objet {\tt TaquinEtat}), afin que vous puissiez
  remonter vers la racine de l'arbre de recherche et retourner
  la solution finale, à la fin de A*.

  Le script Python {\tt taquin.py} importera la fonction {\tt
    joueur\_taquin} contenue dans {\tt solution\_taquin.py} (spécifié via le paramètre {\tt -joueur}) et l'utilisera afin de résoudre le jeu
  du Taquin à partir d'une configuration initiale fixée.


  Voici comment utiliser ce script:
\begin{verbatim}
Usage: python taquin.py [-joueur JOUEUR] [-no_partie INT] [-valider FICHIER]

où -joueur    : "humain" ou le fichier contenant votre solution.
                Défaut = "solution_taquin.py"
   -no_partie : un entier quelconque
                Défaut = partie utilisée pour générer "taquin_validation.pkl"
   -valider   : un fichier permettant de valider votre joueur pour un jeu donné.
                Défaut = "taquin_validation.pkl"

\end{verbatim}

  Par défaut, le jeu est toujours initialisé à la même configuration
  et la solution de votre agent est comparée à la solution attendue,
  contenue dans le fichier {\tt taquin\_validation.pkl} (spécifié via
  le paramètre {\tt -valider}). Le fichier {\tt
    taquin\_validation.txt} contient également une impression de la
  solution attendue. Si vous souhaitez varier la configuration
  initiale afin de tester votre agent sur d'autres configurations,
  changez le numéro de partie via le paramètre {\tt -no\_partie} (un entier
  positif, qui sera utilisé par le générateur aléatoire de
  configuration comme germe ou {\it seed}).  Notez que certaines
  configurations pourraient nécessiter plus de temps pour être
  résolues. Si vous désirez voir comment ce comporte votre agent, utilisez
  l'option {\tt -v} pour faire afficher la grille après chaque coup.

  {\bf Conseils:} dans le cours, on a toujours considéré que la liste
  {\it open} était maintenue ordonnée. En fait, il peut être plus
  simple de seulement s'assurer que le premier élément soit celui
  associé à la valeur de fonction d'évaluation la plus petite. Vous
  pouvez même ne pas ordonner la liste du tout et trouver le minimum
  à chaque fois que vous sortez l'état courant de {\it open}. À vous
  de décider, pourvu que votre choix donne une implémentation qui
  ne soit pas trop lente. Finalement, si vous souhaitez analyser
  l'efficacité de votre code, il suffit d'ajouter {\tt -m cProfile}
  lors de l'exécution (par exemple {\tt python -m cProfile taquin.py}).

\newpage

\item {\bf [5 points]} Programmez un joueur optimal de {\it
    tic-tac-toe} en implémentant l'algorithme de recherche
  alpha-bêta. La recherche alpha-bêta devra donc se faire jusqu'à la
  profondeur maximale.

  Le programme doit être dans le langage Python. Plus spécifiquement,
  vous devez remettre un fichier nommé {\tt solution\_tictactoe.py} contenant une
  fonction nommée {\tt joueur\_tictactoe} prenant 4
  arguments (dans l'ordre):
  \begin{itemize*}
  \item {\tt etat}: Objet {\tt TicTacToeEtat} représentant l'état actuel du jeu.
  \item {\tt fct\_but}: Une fonction d'utilité prenant un état en argument
    et qui retourne l'utilité associée à l'état (grand nombre positif
    si le joueur {\tt 'X'} gagne, grand nombre négatif s'il perd, 0 si
    c'est une partie nulle). Si l'état ne correspond pas à une partie
    terminée, {\tt fct\_but} retourne {\tt None}.
  \item {\tt fct\_transitions}: Une fonction acceptant un état comme
    argument et donnant en sortie un dictionnaire qui associe chaque
    action possible (clé du dictionnaire) à l'état qui en résulte
    (valeur du dictionnaire).
  \item {\tt str\_joueur}: Le rôle joué par ce joueur, c'est-à-dire {\tt
      'X'} ou {\tt 'O'} (sous forme d'une chaîne de caractères).
  \end{itemize*}
  La fonction {\tt joueur\_tictactoe} doit retourner l'action
  prise par votre joueur. Cette action doit donc être une clé valide du dictionnaire
  retourné par {\tt fct\_transitions}.

  Le script Python {\tt tictactoe.py} importera la fonction {\tt
    joueur\_tictactoe} contenue dans \\{\tt
    solution\_tictactoe.py} (qui doit être dans le même répertoire) et
  l'utilisera pour simuler le joueur {\tt 'X'}. Pour tester votre
  code, vous pouvez jouer vous-même le joueur {\tt 'O'}, invoquer un joueur
  aléatoire ou utiliser la même fonction {\tt
    joueur\_tictactoe}. Voici comment utiliser
  {\tt tictactoe.py}:
\begin{verbatim}
Usage: python tictactoe.py [-joueur1 JOUEUR] [-joueur2 JOUEUR]

où -joueur1 : "humain", "aléatoire" ou le fichier contenant votre solution.
              Défaut = "solution_tictactoe.py"
   -joueur2 : "humain", "aléatoire" ou le fichier contenant votre solution.
              Défaut = "solution_tictactoe.py"
\end{verbatim}

  Si vous désirez voir comment ce comporte votre agent, utilisez
  l'option {\tt -v} pour faire afficher la grille après chaque coup.

  La correction de votre code se fera
  de façon automatique, en vérifiant automatiquement l'ensemble des états
  visités par votre recherche alpha-bêta. Puisque l'ensemble des états
  visités dépend de l'ordre dans lequel les états successeurs sont générés,
  on vous demande de {\bf toujours} utiliser l'ordre suggéré par Python
  en itérant sur les items du dictionnaire:

\begin{verbatim}
for action,nouvel_etat in fct_transitions(etat).items():
    # Votre code
    ...

\end{verbatim}

  {\bf Conseil:} assurez-vous que votre joueur puisse jouer {\bf parfaitement}
  les {\tt 'X'} ainsi que les {\tt 'O'}.

\newpage

\item {\bf [BONUS]} Implémentez également un joueur pour le jeu
  Connect~4 (Puissance 4:
  \url{http://fr.wikipedia.org/wiki/Puissance_4}).
  Normalement, votre code pour le tic-tac-toe
  devrait également fonctionner pour Connect~4. Par contre, une
  recherche jusqu'à la profondeur maximale est trop lente dans le cas
  de Connect~4. Ainsi, vous devrez l'ajuster afin de limiter la
  profondeur de la recherche et développer une heuristique pour
  compenser. Plus votre heuristique sera bonne, plus les chances de
  gagner de votre joueur seront bonnes.

  Pour ce faire, vous devez remettre un fichier nommé {\tt solution\_connect4.py}
  contenant une fonction {\tt joueur\_connect4}. Cette
  fonction doit prendre les mêmes arguments que {\tt
    joueur\_tictactoe}, en plus d'un cinquième argument:
  \begin{itemize*}
  \item {\tt int\_tempsMaximal}: le temps maximal, en secondes, accordé
    au joueur pour retourner une action.
  \end{itemize*}
  Si le joueur prend plus qu'une fois et demi (150\%) du temps
  maximal accordé pour choisir une action, il sera éliminé et perdra
  par défaut. S'il prend moins que 150\% du temps, mais plus de 100\%,
  le temps excédant pris par le joueur sera alors retranché au temps
  accordé à son tour suivant.

  Pour tenir compte du temps écoulé, vous pouvez utiliser le module
  standard {\tt time} et sa fonction {\tt time()} qui retourne le
  temps (en secondes) écoulé depuis le $1^{\rm er}$ janvier 1970 (pour
  les systèmes Unix). Plus de détails sur le module {\tt time} sont
  disponibles ici:
  \url{http://docs.python.org/library/time.html}.

  Pour développer votre fonction d'évaluation heuristique, vous
  pourrez évaluer la qualité d'un état à partir du champ {\tt
    etat.tableau} de celui-ci. Ce champ est un tableau Numpy de
  caractères. Il contient l'information sur l'occupation de chaque
  case du jeu, à toutes les positions possibles: chaque entrée vaut
  donc {\tt 'X'}, {\tt 'O'} ou {\tt ' '} (pour une case
  non occupée). Par exemple, l'état du jeu
\begin{verbatim}
| | | | | | | | |
| | | | | | | | |
| |X| | | | | | |
| |X| |O| | | | |
| |X| |O| | | | |
| |X| |O| | |O|X|
-----------------
 0 1 2 3 4 5 6 7
\end{verbatim}
  correspond à la valeur de {\tt etat.tableau} suivante:
\begin{verbatim}
[[' ' ' ' ' ' ' ' ' ' ' ' ' ' ' ']
 [' ' ' ' ' ' ' ' ' ' ' ' ' ' ' ']
 [' ' 'X' ' ' ' ' ' ' ' ' ' ' ' ']
 [' ' 'X' ' ' 'O' ' ' ' ' ' ' ' ']
 [' ' 'X' ' ' 'O' ' ' ' ' ' ' ' ']
 [' ' 'X' ' ' 'O' ' ' ' ' 'O' 'X']]
\end{verbatim}

  L'évaluation de votre solution se fera sous la forme d'un tournoi,
  où tous les joueurs soumis auront à joueur un contre l'autre. {\bf Deux
  points boni} seront accordés aux 5 meilleurs joueurs du
  tournoi. {\bf Un point boni} ira aux 5 suivants. Par contre, pour obtenir
  ces points, les joueurs devront également avoir fait mieux qu'un
  joueur (gardé secret) utilisant une heuristique très simple.

\end{enumerate}


\end{document}
